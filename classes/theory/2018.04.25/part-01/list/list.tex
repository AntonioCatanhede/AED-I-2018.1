% !TeX program = xelatex
\documentclass[a4paper, 12pt]{article}

\usepackage[brazil]{babel}
\let\latinencoding\relax
\usepackage{fontspec}
\usepackage[T1]{fontenc}
\usepackage{lmodern}

\usepackage[a4paper, margin=2cm]{geometry}

\title{Lista de exercícios - Heapsort}
\date{\today}

\begin{document}
  \begin{center}
    {\LARGE Algoritmos e Estruturas de Dados I} \\[1em]
    {\large Lista de Exercícios - Heapsort} \\[1em]
    {\large\today}
  \end{center}

  \vspace{1em}

  \begin{enumerate}
    \item (Cormen 6.2-1) Ilustre passo a passo a operação de
    \textsc{Max-Heapify}$(v,14,2)$ no vetor  \texttt{v} que tem os
     elementos \texttt{\{27, 17, 3, 16, 13, 10, 1, 5, 7, 12, 4, 8, 9, 0\}}.

    \item Implemente uma versão iterativa da função \textsc{Max-Heapify}.

    \item Implemente a função \textsc{Min-Heapify}, onde dado um
    \emph{heap} representado por um vetor \texttt{v}, com tamanho
    \texttt{n}, e um dado nó \texttt{i}, rearranje \texttt{v}
    de modo que o nó \texttt{i} seja um \emph{heap} mínimo.

    \item (Cormen 6.3-1) Ilustre passo a passo a operação de
    \textsc{Build-Max-Heap}$(v,9)$ no vetor \texttt{v} que tem os
    elementos \texttt{\{5, 3, 17, 10, 84, 19, 6, 22, 9\}}.

    \item Implemente a função \textsc{Build-Min-Heap}, que
    recebe como parâmetros de entrada um vetor \texttt{i} com
    tamanho \texttt{n}, e constrói um \emph{heap} mínimo.

    \item Ilustre passo a passo a ordenação \textsc{Heapsort},
    desde a construção do \emph{heap},
    sobre o vetor \texttt{v} que tem os elementos 
    \texttt{\{15, 9, 1, 7, 52, 2, -1, -8, 10, 20, 18, 4, 6, 2\}}.

    \item Implemente uma variante do \textsc{Heapsort}, que tem
    um parâmetro de entrada adicional \texttt{desc}, que caso
    seja verdadeiro, deve ordenar o vetor \texttt{v} de \texttt{n}
    elementos em ordem decrescente, caso contrário, deve ordenar em
    ordem crescente.
  \end{enumerate}
\end{document}